\section{Structure du projet}
\label{sec:project-structure}

Le d\'ep\^ot est organis\'e en dossiers principaux afin de s\'eparer clairement
les ressources, le code C++, les scripts et les livrables. Le sch\'ema
suivant r\'esume l'arborescence \`a haut niveau.

\begin{center}
\begin{verbatim}
HexProject/
|-- assets/
|-- docs/
|   `-- sections/
|-- include/
|   |-- core/
|   |-- gnn/
|   `-- ui/
|-- src/
|   |-- core/
|   |-- gnn/
|   |-- ui/
|   `-- cli/
|-- selfplay/
|   `-- build/ (g\'en\'er\'e)
|-- scripts/
|   `-- models/
|-- rapport_hex/
|-- CMakeLists.txt
|-- README.md
|-- pyproject.toml
`-- poetry.lock
\end{verbatim}
\end{center}

\begin{table}[h]
\centering
\begin{tabular}{ll}
\hline
Dossier & Contenu et r\^ole g\'en\'eral \\
\hline
assets/ & images, sons et textures pour l'interface \\
docs/ & sources LaTeX du rapport final et sections \\
include/ & en-t\^etes C++ (API publique par module) \\
src/ & impl\'ementation C++ (logique, GNN, UI, CLI) \\
selfplay/ & g\'en\'eration de parties et entra\^inement auto-jeu \\
scripts/ & scripts Python et artefacts de mod\`eles \\
rapport\_hex/ & version IEEE du premier rapport \\
\hline
\end{tabular}
\caption{R\'epartition des dossiers principaux.}
\end{table}

Les fichiers racine (\texttt{CMakeLists.txt}, \texttt{README.md},
\texttt{pyproject.toml}, \texttt{poetry.lock}) rassemblent la configuration
de build, la documentation et les d\'ependances Python.