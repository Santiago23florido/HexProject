\subsection{Gestion des Ressources Multimedia et Système Audio}
Cette sous-section aborde la stratégie d'optimisation de la mémoire pour les actifs (\textit{assets}) et l'implémentation du moteur sonore interactif.

\subsubsection{Optimisation de la Mémoire et Chargement des Assets}
Pour garantir une performance stable à 60 FPS, l'interface utilise une stratégie de pré-chargement centralisée dans le constructeur de \texttt{HexGameUI}. 
\begin{itemize}
    \item \textbf{Gestion des Textures :} Les textures sont chargées une seule fois en mémoire vive (RAM/VRAM). Les instances de \texttt{HexTile} ne stockent que des références ou des pointeurs vers ces ressources, appliquant ainsi une variante du patron de conception \textit{Flyweight}.
    \item \textbf{Robustesse du Système :} L'utilisation de \texttt{std::filesystem} assure la compatibilité des chemins d'accès entre différents systèmes d'exploitation (Linux/Windows), tandis que des blocs \texttt{try-catch} préviennent les pannes en cas de fichiers manquants.
\end{itemize}

\subsubsection{Architecture du Contrôle Audio Dynamique}
Le système sonore est divisé en trois flux indépendants, permettant une personnalisation complète de l'expérience utilisateur. Le Tableau \ref{tab:audio_config} détaille les paramètres de persistance configurés dans le logiciel.

\begin{table}[h]
\renewcommand{\arraystretch}{1.3}
\caption{Configuration des Canaux Audio}
\label{tab:audio_config}
\centering
\begin{tabular}{|l|c|p{3.5cm}|}
\hline
\textbf{Paramètre} & \textbf{Type SFML} & \textbf{Usage} \\ \hline
Musique Menu & \texttt{sf::Music} & Streaming continu pour l'ambiance. \\ \hline
Effets (SFX) & \texttt{sf::Sound} & Lecture immédiate depuis un \textit{buffer}. \\ \hline
Volume Master & \texttt{float} & Facteur multiplicateur global (0-100). \\ \hline
\end{tabular}
\end{table}

\subsubsection{Logique de Mise à Jour de l'Iconographie Sonore}
La fonction \texttt{updateVolumeIcon} est un exemple d'interaction dynamique. Elle traduit une valeur scalaire (le volume) en un état visuel discret parmi quatre paliers :
\begin{equation}
    Icon_{index} =
    \left\{
    \begin{array}{ll}
    0 & \mbox{si } v \le 25 \\
    1 & \mbox{si } 25 < v \le 50 \\
    2 & \mbox{si } 50 < v \le 75 \\
    3 & \mbox{si } v > 75
    \end{array}
    \right.
\end{equation}
Cette logique permet de changer le \textit{sprite} de l'icône de haut-parleur en temps réel lors de la manipulation des \textit{sliders}, offrant un retour visuel immédiat sur la configuration du système.



\subsubsection{Persistance des Paramètres}
Le logiciel inclut une routine de sauvegarde et chargement automatique des niveaux de volume. Lors de l'initialisation, le système tente de lire un fichier de configuration ; en cas d'erreur ou d'absence, il restaure des valeurs par défaut sécurisées, garantissant que l'application reste toujours dans un état stable et fonctionnel pour l'utilisateur.
