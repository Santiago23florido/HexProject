\section{Installation}
\label{sec:installation}

\subsection{Clonage du depot}
\begin{verbatim}
git clone https://github.com/\
Santiago23florido/HexProject.git
cd HexProject
\end{verbatim}

\subsection{Pr\'erequis}
\paragraph{CMake}
\begin{verbatim}
cmake --version
\end{verbatim}

\paragraph{Compilateur C++17}
\begin{verbatim}
g++ --version
# ou: clang++ --version
\end{verbatim}

\paragraph{LibTorch (TorchScript)}
\begin{verbatim}
ls $HOME/libtorch/share/cmake/Torch
# ou definir Torch_DIR/CMAKE_PREFIX_PATH
\end{verbatim}

\subsection{Installation de LibTorch}
\begin{itemize}
\item T\'el\'echarger la distribution C++ (LibTorch) depuis le site officiel PyTorch.
\item Choisir l'ABI C++ adapte :
  \begin{itemize}
  \item Linux GCC/Clang : pr\'ef\'erer la variante \texttt{cxx11 ABI}.
  \item Windows : choisir le binaire correspondant a MSVC.
  \end{itemize}
\item Choisir CPU ou CUDA selon votre environnement.
\item D\'ecompresser, par exemple dans \texttt{\$HOME/libtorch}.
\end{itemize}

\paragraph{T\'el\'echargement (CPU par d\'efaut)}
\begin{verbatim}
wget https://download.pytorch.org/libtorch/\
nightly/cpu/libtorch-shared-with-deps-latest.zip \
  -O libtorch-cpu.zip
unzip libtorch-cpu.zip
mv libtorch $HOME/libtorch
\end{verbatim}
Pour CUDA, t\'el\'echarger la version compatible avec votre toolkit depuis
\texttt{https://pytorch.org/get-started/locally/}.

\paragraph{Variables de configuration}
\begin{verbatim}
export Torch_DIR="$HOME/libtorch/share/cmake/Torch"
# ou
export CMAKE_PREFIX_PATH="$HOME/libtorch"
\end{verbatim}

\paragraph{Configuration CMake}
\begin{verbatim}
cmake -S . -B build \
  -DCMAKE_PREFIX_PATH="$HOME/libtorch"
\end{verbatim}

\paragraph{Problemes de chargement a l'execution}
\begin{verbatim}
export LD_LIBRARY_PATH="$HOME/libtorch/lib:\
$LD_LIBRARY_PATH"
\end{verbatim}

\paragraph{SFML 2.5}
\begin{verbatim}
pkg-config --modversion sfml-all
# ou: sudo apt-get install libsfml-dev
\end{verbatim}

\paragraph{CUDA Toolkit (optionnel)}
\begin{verbatim}
nvcc --version
\end{verbatim}

\paragraph{Python 3 + poetry (optionnel)}
\begin{verbatim}
python3 --version
poetry --version
\end{verbatim}

\subsection{Si un pr\'erequis manque}
\begin{itemize}
\item \textbf{CMake} : installer via le gestionnaire de paquets ou les binaires officiels.\\
Ex. Debian/Ubuntu : \texttt{sudo apt-get install cmake}.
\item \textbf{Compilateur C++} : installer GCC/Clang (ou MSVC sous Windows) avec support C++17.\\
Ex. Debian/Ubuntu : \texttt{sudo apt-get install g++} ou \texttt{sudo apt-get install clang}.
\item \textbf{LibTorch} : t\'el\'echarger la distribution C++ adapt\'ee a votre OS/ABI.\\
Puis d\'efinir \texttt{Torch\_DIR} ou \texttt{CMAKE\_PREFIX\_PATH}.
\item \textbf{SFML} : installer les paquets de d\'eveloppement.\\
Ex. Debian/Ubuntu : \texttt{sudo apt-get install libsfml-dev}.
\item \textbf{CUDA} (optionnel) : installer un toolkit correspondant a la build LibTorch,\\
sinon utiliser une build CPU.
\item \textbf{Python + poetry} (optionnel) : installer Python 3 et poetry.\\
Ex. Debian/Ubuntu : \texttt{sudo apt-get install python3 python3-pip}.\\
Poetry : \texttt{python3 -m pip install --user poetry}.
\end{itemize}

\subsection{Compatibilit\'e des versions}
\begin{itemize}
\item CMake : 3.18+ (requis par CMakeLists.txt).
\item Compilateur C++ : support complet de C++17 (ex. GCC 7+, Clang 6+, MSVC 2019+).
\item SFML : 2.5+ (requis par find\_package(SFML 2.5)).
\item LibTorch : distribution C++ correspondant a l'OS et a l'ABI du compilateur ; si CUDA, version CUDA compatible avec la build LibTorch.
\end{itemize}
Si une version n'est pas compatible, mettre a jour vers la version minimale ci-dessus ou choisir un binaire correspondant.

\subsection{Installation par OS}

\subsubsection{Linux}
\paragraph{Ajout au PATH (si n\'ecessaire)}
\begin{verbatim}
# ~/.bashrc
# poetry
export PATH="$HOME/.local/bin:$PATH"
# cmake (archive)
export PATH="/opt/cmake/bin:$PATH"
# cuda
export PATH="/usr/local/cuda/bin:$PATH"
export LD_LIBRARY_PATH="/usr/local/cuda/lib64:\
$LD_LIBRARY_PATH"
export LD_LIBRARY_PATH="$HOME/libtorch/lib:\
$LD_LIBRARY_PATH"
source ~/.bashrc
\end{verbatim}

\paragraph{Compilation et execution du jeu}
\begin{verbatim}
rm -rf build
cmake -S . -B build
cmake --build build
cd build
./hex_ui
\end{verbatim}

\paragraph{Compilation et execution du selfplay}
\begin{verbatim}
rm -rf selfplay/build
cmake -S selfplay -B selfplay/build
cmake --build selfplay/build
./selfplay/build/selfplay \
  --selfplay-train \
  --export-ts \
  --train-games 200 \
  --min-depth 10 \
  --max-depth 20 \
  --batch-size 256 \
  --updates-per-game 1 \
  --device cuda
\end{verbatim}

\subsubsection{Windows}
\paragraph{Pr\'erequis}
\begin{itemize}
\item CMake 3.18+.
\item MSVC (Visual Studio 2019+).
\item SFML 2.5+ (d\'efinir \texttt{SFML\_DIR} ou \texttt{SFML\_BIN\_DIR}).
\item Inno Setup (\texttt{iscc.exe} dans le PATH) pour g\'en\'erer l'installateur.
\end{itemize}

\paragraph{Build + installateur (CPU)}
\begin{verbatim}
# PowerShell depuis la racine du repo
# optionnel : definir LIBTORCH_DIR
$env:SFML_DIR="C:\path\to\SFML"
# optionnel : $env:SFML_BIN_DIR="C:\path\to\SFML\bin"
.\scripts\package_windows_cpu.ps1
\end{verbatim}
Le script t\'el\'echarge LibTorch CPU si besoin, compile en Release,
pr\'epare les assets, copie les DLL et lance Inno Setup.

\subsection{Fast-installation (Linux, gameplay only)}
\paragraph{WSL (Ubuntu)}
Requiert WSL avec Ubuntu 24.04.3 LTS ou sup\'erieur.
V\'erifier avec \texttt{lsb\_release -a}.
\paragraph{Telechargement}
\begin{verbatim}
sudo apt-get update
sudo apt-get install -y python3 python3-venv \
  python3-pip
python3 -m venv .venv
source .venv/bin/activate
pip install gdown
gdown --fuzzy \
  "https://drive.google.com/file/d/"\
"1iVbLBNMKzCeTfzk1X_cW3bPPNIjGCMX-/view?usp=sharing" \
  -O HexProject-CPU-x86_64.AppImage
\end{verbatim}
\paragraph{Verification}
\begin{verbatim}
file HexProject-CPU-x86_64.AppImage
\end{verbatim}
\paragraph{Execution}
\begin{verbatim}
chmod +x HexProject-CPU-x86_64.AppImage
./HexProject-CPU-x86_64.AppImage
\end{verbatim}
\paragraph{Audio (WSL)}
Si vous avez des probl\`emes de son sous WSL, ex\'ecuter :
\begin{verbatim}
sudo apt update && sudo apt install -y \
  libasound2t64 \
  libopenal1 \
  pulseaudio \
  libvorbisfile3 \
  libsndfile1
\end{verbatim}
Puis relancer le jeu.

\paragraph{Windows (gameplay only)}
Ouvrir le lien, t\'el\'echarger, puis lancer
l'installateur.
\begin{verbatim}
https://drive.google.com/file/d/1XExLIMiUIn2Q0FxtC6UjaX-f9eCW9wgK/view?usp=drive_link
\end{verbatim}
